A company's economical value needs to be assessed for use cases like asset investment, speculative trading of stocks, estimating a company's credit risk or making internal business decision in consideration of competitors, suppliers and business partners. In the context of market analysis and financial forecasts, the economical value of a stock is called the intrinsic value.

% Add to problem definition: The stock market is driven by events and emotions. It will be investigated if there are underlying structures between stock prices which can be observed by variables extracted from financial news and stock price.

As implied by the theory of Bounded Rationality, the intrinsic value of a company's stock is likely not fully reflected by its actual stock price. Business relationships are assumed to partially explain this gap between intrinsic and actual value of stock prices. Concluding, the goal of this work is to \textbf{improve the assessment of a company's intrinsic value by taking its business relationships into account}.

In contrast to their formal definition, this works treats business relationships in a wider context as from the perspective from a single company. Every negatively or positively connotated connections between two companies is considered as being part of their business relationship. Thus, a related company can be any type of company having an direct or indirect influence on the success or failure of another company's business. For example, one company can have an indirect impact on another one by being its competitor.

However, neither the intrinsic stock value nor the business relationships can be determined objectively by some measures. If a business relationship between two companies is present, both stock prices are assumed to evolve similarly. In order to evaluate the above suggested impact of underlying business relationships on a stock's intrinsic value, the impact on cross-correlations between two stock prices will be examined respectively. Subsequently, if two stock prices reveal a similar evolution, it is assumed to be, at least partially, caused by an underlying business relationship between the two affiliated companies.

Last but not least, a business relationship is an abstract property which encompasses a set of various uni- and bidirectional relations, as defined before. Due to their branched nature, the complex composition of relations is not directly observable, news are considered to be an appropriate proxy for this information. Furthermore, financial news in particular will be used because they are \enquote{deemed to have less noise compared with general news}, as stated by \citet{KhadjehNassirtoussi2014TextReview}. Hence, it is assumed that business journalists associate two companies with each other in their financial reporting if they are related in some sense.


% Eqn.~\ref{formula:hypothesis}:

% \begin{align}
%     \begin{split}\label{formula:hypothesis}
%         H_0 &= \text{There is no statistical significant correlation between stock price cross-correlations and business relationships} \\
%         H_1 &= \text{There is a statistical significant correlation between stock price cross-correlations and business relationships}
%     \end{split} \\ \eqname{Null and Alternative Hypothesis}
% \end{align}


% https://en.wikipedia.org/wiki/Fundamental_analysis

% "The paper is organized as follows. In sect. 2 we present our results for thermodynamic quantities. Correlation length measurements are presented and discussed in sect. 3. A summary is given in sect. 4."
% https://drive.google.com/file/d/1GMD2dBogilx3JVeUiRyek3yz8IlJ6C1g/view
% https://sci-hub.se/10.1080/14697688.2015.1070960
% https://sci-hub.se/10.1016/j.eswa.2014.07.040
% https://sci-hub.se/10.1016/j.eswa.2014.08.004