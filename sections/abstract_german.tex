\section*{Zusammenfassung}

Aufgrund der vielfältigen Einflüsse ist es schwierig, einzelne Ursachen für Aktienkursbewegungen zu identifizieren. Bisherige Studien untersuchen in Modellen überwiegend äußere Einflüsse, wie z.B. Nachrichtenartikel. Diese Arbeit hebt hervor, dass die Geschäftsbeziehungen zwischen Unternehmen ebenso mit einbezogen werden müssen, um ein besseres Verständnis von der ähnlichen Entwicklung von Aktienkursen zu erlangen. Zu diesem Zweck werden statistische und ökonometrische Methoden verwendet, um Aktienkurse von 467 Unternehmen aus dem Marktindex S\&P~500 statistisch zu bereinigen und miteinander zu korrelieren. Desweiteren werden repräsentative Variablen für Geschäftsbeziehung aus Finanzartikeln von Reuters und Bloomberg extrahiert. Die abschließende Korrelationsanalyse zwischen den beiden entstandenen Variablen, Ähnlichkeit von Aktienkursen und Geschäftsbeziehungen, zeigt eine signifikante Beziehungen zwischen beiden auf. Dieses Ergebnis deutet darauf hin, dass Aktienkurse nicht einzeln, sondern zusammen mit verwandten Aktienkursen modelliert werden sollten.
Zukünftige Arbeiten sollten Graph-Modelle miteinbeziehen, um besser zu verstehen, welche Faktoren die Bewegungen von Aktienkursen beeinflussen.


% Bewegungen von Aktienkursen sind schwer nachzuvollziehen. Um zu verstehen, was sie verursacht und wie man sie sogar voraussagen kann, beziehen bisherige Studien vorallem äußere Einflüße, wie

%  am Beispiel der Firma Apple

% Sprachlich nüchtern abholen

% Was sind die wichtigsten Ergebnisse? Welche Methodik wurde wie angewendet? Was sind die wichtigsten Schlussfolgerungen usw.?

% https://studi-lektor.de/tipps/bachelor-thesis/abstract-schreiben.html


% statistische/quantitative Untersuchung -> begründe warum diesen ansatz