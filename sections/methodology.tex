In order to investigate the impact of business relationships on the correlation of stocks, features need to be extracted at first. In the following, the methods used for feature extraction and evaluation will be presented.

\paragraph{Stock Correlation}
For measuring the correlation among two stocks, the correlation coefficient among their historical prices over a period of four years will be calculated. Because economic variables, and thus stock prices, are considered to be unit root processes \cite{Granger1974SpuriousEconometrics}, they are very likely to falsely reveal mutual correlation, also known as spurious correlation. Regressing on such variables therefore leads to spurious regression \cite{Yule1926WhyTime-Series}. To filter out potential causes for this error, the data needs to be prewhitened \cite{Dean2016DangersModels} and statistical conditions for correlation and regression need to be assured. According to \citet{Granger1974SpuriousEconometrics}, one should not expect the data to be completely free from spurious correlations, even though the aforementioned step were applied: \enquote{Although any of these methods will undoubtedly alleviate the problem in general, it is doubtful if they will completely remove it}. Because a large set of stock prices will be used throughout this work, the results cannot be individually checked or visually evaluated. Instead, a diverse set of hypothesis tests will be applied to find evidence for statistical properties like stationarity. After reducing potential causes for spurious correlation and thereby ensuring preconditions for correlation testing, the Pearson product-moment correlation coefficient, denoted as $r$, will be calculated.

\paragraph{Co-occurrence}
Besides the stock correlations, a second feature needs to be introduced which represents the business relationship. As stated previously in Section~\ref{section:problem_definition}, financial news are used therefore. They are known to be strongly linked to the market in both directions, i.e. they not only report market status but they actively create an impact on investors behaviour \cite{KhadjehNassirtoussi2014TextReview}. Hence, they are considered to be an appropriate source of information for finding business relationships. Thereby, the simultaneous occurrence of two companies in an article is considered to be representative for their relationship. The joint appearance will be called co-occurrence in the following. In order to identify occurrences of companies, Named Entity Recognition (NER) will be applied. NER is a method of information extraction which locates sequences of words in an unstructured text and labels them by predefined general categories, such as \enquote{person}, \enquote{location} or \enquote{organization}. To filter and link found organization entities to historical prices, Named Entity Linking (NEL) will be used, which seeks to link classified entities to entries from a selected knowledge base. The used knowledge base will contain historical stock prices, ticker symbols (e.g. \emph{MSFT}) and official company names (e.g. \emph{Microsoft Corp.}) which can be used for NEL. Based on the extracted and linked occurrences, three handcrafted features with differing emphasis on either frequency or distance of co-occurrences will be introduced in Section~\ref{section:text_analysis}.

\paragraph{Evaluation}
In order to assess the success of feature extraction and to evaluate my hypothesis, the correlation between the two extracted features stock correlation and co-occurrence is calculated and discussed. To compare the three co-occurrence based features with each other, they will be correlated separately with the stock correlation feature. The most meaningful feature is expected to reveal the highest correlation. Subsequently, the most promising news-based feature resulting from this experiment is used to retrospectively assess the prewhitening steps for stock prices. The stock correlation will be calculated for every intermediate preprocessing step in order to evaluate if they are supportive for a higher correlation with co-occurrence. Finally, my hypothesis is evaluated by the overall observed correlation coefficients between both resulting features, extracted from stock prices and financial news. To ensure the coefficients significance, the confidence intervals will be provided.

% Any attempt to evaluate the results would be premature at this stage, since some of these initiatives require more time for concrete and tangible results to be reached.  % http://www.forschungsgesellschaft.at/rereal/TOWARDS%20A%20FRAMEWORK%20IN%20OFL.PDF

\paragraph{Limitations}
The financial markets are prone to a vast number of external influences which cannot be all collected. Especially, political tension and other events can have a huge impact but are not easy collectible because they might be missing in financial reporting. Further, financial markets across the world reveal different dynamics \cite{Hsu2016BridgingEconomists} and need to be examined separately with suitable data. For those reasons, some problem limitations are defined in the following: 

\begin{itemize}
    \item Only the New York Stock Exchange (NYSE) and the NASDAQ Exchange will be considered. Measured upon market capitalization and number of listed companies, NYSE and NASDAQ are the world's largest exchanges. Consequently, findings based on these might be supported for other great stock exchanges, such as Tokyo Stock Exchange and Shanghai Stock Exchange, but need to be examined by an appropriate dataset.
    \item Both exchanges contain more than 3000 companies each. Because small companies are less likely to be mentioned in financial news, only components from the S\&P~500 market index are used which are considered to be the most important ones for the two exchanges.
    \item Business relations are not steady, but fluctuate and vanish and new ones emerge due to an altering corporate policy. This problem is not treated and the relationships are considered to be steady for the examined period. Subsequently, the stock correlations and news-based co-occurrences for the whole period are collected, each one condensed to one coefficient per pair of companies.
\end{itemize}

In practice investment corporations are optimizing their prediction models with every bit of information which might be useful. The markets behavior will always adapt to new trading algorithms and thereby its market efficiency vanishes such improvements. Hence, investors try to keep their intellectual capital saved and usually do not share successful autonomous trading methods - to the chagrin of the scientific community. Therefore, it is challenging for scientists to develop a system which accommodates real-world constraints \cite{Vanstone2009AnNetworks}. Also, a high degree of random factors remains, among others, because of the offer-demand-fluctuations. With obstacles as the previously mentioned ones in mind, this work does not aim for a proof of the hypothesis that business relationship are reflected in stock prices. Rather, evidence is collected in order to provide a potentially new direction of interest for research in market analysis and stock price prediction.

% \item Trading companies usually apply fast simple models for short-term predictions
% \emph{Given that the structure of an econometric model consists of optimal decision rules of economic agents, and that optimal decision rules vary systematically with changes in the structure of series relevant to the decision maker, it follows that any change in policy will systematically alter the structure of econometric models. (Lucas Critique, Philipps Curve)}
% \todo{Mention Lucas Critique}


% Let $t-1$ be the last day and $t$ be the first day in the future, the initial models input should be the historical stock prices of the target corporation over the last $D_P$ days $\{t-D_P, ..., t-1\}$. The stock prices for day $t$ are represented by opening price $p_{open}(t)$, day high $p_{high}(t)$, day low $p_{low}(t)$ and closing price $p_{close}(t)$. The model should then predict the direction towards which the price will change over the next $D_F$ days $\{t, ..., t+D_F-1\}$.

% As pointed out by \citet{Tsantekidis2017UsingMarkets}, only two classes $up$ and $down$ would introduce noise to the prediction labels, since the smallest changes would be considered as a upward or downward movement. To handle these cases a third class $still$ is introduced which represents no movement over the next $D_F$ days. A threshold $\epsilon$ will be defined which determines the margins of this third class. The label $y_t$ is then defined as follows:
% \begin{equation}
% 	y_t = \begin{cases}
%         1, & \text{if } p_{close}(t+D_F-1) > p_{open}(t) \cdot (1 + \epsilon) \\
%         -1,& \text{if } p_{close}(t+D_F-1) < p_{open}(t) \cdot (1 - \epsilon) \\
%         0, & \text{otherwise}
%     \end{cases}
% \end{equation}

% Based on this initial setup the model will be enhanced step by step with fundamental data of the target and its related corporations, the historical price of related corporations and their relationships to the target corporation. This graph data might be fed into the model as a matrix of feature vectors representing the graphs edges (corporate relationships), another matrix of feature vectors representing the nodes (general information, e.g. annual sales) and a third matrix for the historical stock prices of all considered corporations.

% The challenge of this work will be to generate an appropriate corporate network and build a prediction model exploiting this knowledge. Similar data is used as starting point as in the related work. The main emphasis is on transforming the raw text data into a reasonable structure format instead of introducing a better DNN for automatically extracting more meaningful relations.